\documentclass[]{article}
\usepackage{graphicx}
\usepackage[a4paper, top=2.5cm, bottom=2.5cm, left=3cm, right=3cm]{geometry}
\usepackage[hidelinks]{hyperref}

%title
\title{Practica 1} 

\author{Adrián Ferández Galán y César López Mantecón}

\begin{document}

\begin{titlepage}
    \centering
   \includegraphics[width=0.9\textwidth]{uc3m.jpg} 
    {\Huge Universidad Carlos III\\
    
     \Large Sistemas Distribuídos\\
     \vspace{0.5cm}
     Curso 2023-24}
    \vspace{2cm}

    {\Huge \textbf{Práctica 1} \par}
    \vspace{0.5cm}
    {\Large Colas de mensajes POSIX \par}
    \vspace{8cm}

   \textbf{Ingeniería Informática, Tercer curso}\\
    \vspace{0.2cm} 
    Adrián Fernández Galán (NIA: 100472182, e-mail: 100472182@alumnos.uc3m.es) \\
    César López Mantecón   (NIA: 100472092, e-mail: 100472092@alumnos.uc3m.es)
    \vspace{0.5cm}

   
    \textbf{Prof .} Félix García Caballeira\\
    \textbf{Grupo: } 81   
    
\end{titlepage}
\newpage

\renewcommand{\contentsname}{\centering Índice}
\tableofcontents

\newpage
\section{Introducción}
\label{sec:introduccion}
Tu contenido de la sección de introducción.

\section{Diseño}
\label{sec:disenno}
Tu contenido de la sección de desarrollo.
\subsection{Mensajes}
\label{subsec:mensajes}
Aquí explicar los argumentos de entrada y salida que nos llevan a desarrollar las estructuras que hemos usado.


\subsection{Uso de ficheros}
\label{subsec:uso_de_ficheros}
Aquí explicar cómo hemos usado ficheros como forma de almacenamiento y por qué. Comentar la escritura atómica para buffers pequeños. 


\subsection{Servidor}
\label{subsec:servidor}
Aquí explicar el diseño concurrente del servidor. Donde están los mutex y por qué. Diseño de pool de hilos o hilo por petición. 

\section{Descripción de pruebas}
\label{sec:descripcion_de_pruebas}
Pruebas que hemos ejecutado sobre nuestra aplicación: ejecución con varios clientes, validación de las funciones... Creo que lo mejor sería dividirlo en subsecciones, una para cada función, otra para las comunicaciones y otra para la concurrencia. 



\section{Conclusiones}
\label{sec:conclusiones}
Escribimos aquí


\end{document}
